\documentclass[11pt,preprint, authoryear]{elsarticle}

\usepackage{lmodern}
%%%% My spacing
\usepackage{setspace}
\setstretch{1.2}
\DeclareMathSizes{12}{14}{10}{10}

% Wrap around which gives all figures included the [H] command, or places it "here". This can be tedious to code in Rmarkdown.
\usepackage{float}
\let\origfigure\figure
\let\endorigfigure\endfigure
\renewenvironment{figure}[1][2] {
    \expandafter\origfigure\expandafter[H]
} {
    \endorigfigure
}

\let\origtable\table
\let\endorigtable\endtable
\renewenvironment{table}[1][2] {
    \expandafter\origtable\expandafter[H]
} {
    \endorigtable
}


\usepackage{ifxetex,ifluatex}
\usepackage{fixltx2e} % provides \textsubscript
\ifnum 0\ifxetex 1\fi\ifluatex 1\fi=0 % if pdftex
  \usepackage[T1]{fontenc}
  \usepackage[utf8]{inputenc}
\else % if luatex or xelatex
  \ifxetex
    \usepackage{mathspec}
    \usepackage{xltxtra,xunicode}
  \else
    \usepackage{fontspec}
  \fi
  \defaultfontfeatures{Mapping=tex-text,Scale=MatchLowercase}
  \newcommand{\euro}{€}
\fi

\usepackage{amssymb, amsmath, amsthm, amsfonts}

\def\bibsection{\section*{References}} %%% Make "References" appear before bibliography


\usepackage[round]{natbib}

\usepackage{longtable}
\usepackage[margin=2.3cm,bottom=2cm,top=2.5cm, includefoot]{geometry}
\usepackage{fancyhdr}
\usepackage[bottom, hang, flushmargin]{footmisc}
\usepackage{graphicx}
\numberwithin{equation}{section}
\numberwithin{figure}{section}
\numberwithin{table}{section}
\setlength{\parindent}{0cm}
\setlength{\parskip}{1.3ex plus 0.5ex minus 0.3ex}
\usepackage{textcomp}
\renewcommand{\headrulewidth}{0.2pt}
\renewcommand{\footrulewidth}{0.3pt}

\usepackage{array}
\newcolumntype{x}[1]{>{\centering\arraybackslash\hspace{0pt}}p{#1}}

%%%%  Remove the "preprint submitted to" part. Don't worry about this either, it just looks better without it:
\makeatletter
\def\ps@pprintTitle{%
  \let\@oddhead\@empty
  \let\@evenhead\@empty
  \let\@oddfoot\@empty
  \let\@evenfoot\@oddfoot
}
\makeatother

 \def\tightlist{} % This allows for subbullets!

\usepackage{hyperref}
\hypersetup{breaklinks=true,
            bookmarks=true,
            colorlinks=true,
            citecolor=blue,
            urlcolor=blue,
            linkcolor=blue,
            pdfborder={0 0 0}}


% The following packages allow huxtable to work:
\usepackage{siunitx}
\usepackage{multirow}
\usepackage{hhline}
\usepackage{calc}
\usepackage{tabularx}
\usepackage{booktabs}
\usepackage{caption}


\newenvironment{columns}[1][]{}{}

\newenvironment{column}[1]{\begin{minipage}{#1}\ignorespaces}{%
\end{minipage}
\ifhmode\unskip\fi
\aftergroup\useignorespacesandallpars}

\def\useignorespacesandallpars#1\ignorespaces\fi{%
#1\fi\ignorespacesandallpars}

\makeatletter
\def\ignorespacesandallpars{%
  \@ifnextchar\par
    {\expandafter\ignorespacesandallpars\@gobble}%
    {}%
}
\makeatother

\newlength{\cslhangindent}
\setlength{\cslhangindent}{1.5em}
\newenvironment{CSLReferences}%
  {\setlength{\parindent}{0pt}%
  \everypar{\setlength{\hangindent}{\cslhangindent}}\ignorespaces}%
  {\par}


\urlstyle{same}  % don't use monospace font for urls
\setlength{\parindent}{0pt}
\setlength{\parskip}{6pt plus 2pt minus 1pt}
\setlength{\emergencystretch}{3em}  % prevent overfull lines
\setcounter{secnumdepth}{5}

%%% Use protect on footnotes to avoid problems with footnotes in titles
\let\rmarkdownfootnote\footnote%
\def\footnote{\protect\rmarkdownfootnote}
\IfFileExists{upquote.sty}{\usepackage{upquote}}{}

%%% Include extra packages specified by user

%%% Hard setting column skips for reports - this ensures greater consistency and control over the length settings in the document.
%% page layout
%% paragraphs
\setlength{\baselineskip}{12pt plus 0pt minus 0pt}
\setlength{\parskip}{12pt plus 0pt minus 0pt}
\setlength{\parindent}{0pt plus 0pt minus 0pt}
%% floats
\setlength{\floatsep}{12pt plus 0 pt minus 0pt}
\setlength{\textfloatsep}{20pt plus 0pt minus 0pt}
\setlength{\intextsep}{14pt plus 0pt minus 0pt}
\setlength{\dbltextfloatsep}{20pt plus 0pt minus 0pt}
\setlength{\dblfloatsep}{14pt plus 0pt minus 0pt}
%% maths
\setlength{\abovedisplayskip}{12pt plus 0pt minus 0pt}
\setlength{\belowdisplayskip}{12pt plus 0pt minus 0pt}
%% lists
\setlength{\topsep}{10pt plus 0pt minus 0pt}
\setlength{\partopsep}{3pt plus 0pt minus 0pt}
\setlength{\itemsep}{5pt plus 0pt minus 0pt}
\setlength{\labelsep}{8mm plus 0mm minus 0mm}
\setlength{\parsep}{\the\parskip}
\setlength{\listparindent}{\the\parindent}
%% verbatim
\setlength{\fboxsep}{5pt plus 0pt minus 0pt}



\begin{document}



\begin{frontmatter}  %

\title{Night Lights and Noisy Data - Using Machine Learning to Better
Detect Human-Generated Light}

% Set to FALSE if wanting to remove title (for submission)




\author[Add1]{Johannes Coetsee}
\ead{19491050@sun.ac.za}





\address[Add1]{Stellenbosch University}


\begin{abstract}
\small{
Abstract to be written here. The abstract should not be too long and
should provide the reader with a good understanding what you are writing
about. Academic papers are not like novels where you keep the reader in
suspense. To be effective in getting others to read your paper, be as
open and concise about your findings here as possible. Ideally, upon
reading your abstract, the reader should feel he / she must read your
paper in entirety.
}
\end{abstract}

\vspace{1cm}

\begin{keyword}
\footnotesize{
Remote Sensing \sep Night Lights \sep Random Forest \\ \vspace{0.3cm}
}
\end{keyword}
\vspace{0.5cm}
\end{frontmatter}



%________________________
% Header and Footers
%%%%%%%%%%%%%%%%%%%%%%%%%%%%%%%%%
\pagestyle{fancy}
\chead{}
\rhead{Advanced Cross Section - January 2022}
\lfoot{}
\rfoot{\footnotesize Page \thepage}
\lhead{}
%\rfoot{\footnotesize Page \thepage } % "e.g. Page 2"
\cfoot{}

%\setlength\headheight{30pt}
%%%%%%%%%%%%%%%%%%%%%%%%%%%%%%%%%
%________________________

\headsep 35pt % So that header does not go over title




\hypertarget{introduction}{%
\section{\texorpdfstring{Introduction
\label{Introduction}}{Introduction }}\label{introduction}}

The use of remote sensing data, and more specifically, satellite
nighttime light data, presents potential for new and diverse
applications in socioeconomic research. Nightlight data remains a
largely objective measure, and is thereby suitable to use as a proxy in
a broad array of studies that require the usage of potentially
unreliable or otherwise lacking data. This advantage is especially
pertinent in parts of the developing world, where socioeconomic research
can prove most beneficial. There exists different night lights products
which can be utilized towards this end, the most common of which is the
`Stable Lights' product, derived from the Defense Meteorological
Satellite Program's (DMSPs) Operational Linescan System (OLS). This
paper emphasises the usage of this product specifically focusing on its
shortcomings. Most prominently, DMSP-OLS has difficulty in separating
background noise from night lights generated from human-generated light,
especially in areas that display lower levels of night light intensity.
This presents an obvious problem: analyses that attempt to use Stable
Lights as a proxy for economic activity, for instance, would exaggerate
or understate economic activity in these low-luminous areas.

This paper attempts to address the challenge of inaccurate measurement
of night lights by applying a filtering technique to identify and
separate nightlights emitted by humans from those emitted by anything
else. This filtering process is based on the methodology for deriving
the `Local Human Lights' product by Määttä \& Lessmann
(\protect\hyperlink{ref-maatta}{2019}), and relies on a Random Forest
(RF) Machine Learning algorithm for classification.

\hypertarget{explication-of-the-problem}{%
\section{Explication of the Problem}\label{explication-of-the-problem}}

\hypertarget{stable-lights-and-economic-activity}{%
\subsection{Stable Lights and Economic
Activity}\label{stable-lights-and-economic-activity}}

The most prominent difficulty, however, relates to the amount of noise
in the lower end of the light distribution due in part to the blooming
effect mentioned above. Standard practice using the stable lights data
set is to discard these values from analysis, thereby removing a large
proportion of cell observations.

DMSP-OLS

\hypertarget{problems-with-stable-lights}{%
\subsection{Problems with Stable
Lights}\label{problems-with-stable-lights}}

\hypertarget{method-and-data}{%
\section{\texorpdfstring{Method and Data
\label{Methodology}}{Method and Data }}\label{method-and-data}}

The filtering process used by Määttä \& Lessmann
(\protect\hyperlink{ref-maatta}{2019}) necessitates the following steps:

\begin{itemize}
\tightlist
\item
\item
\item
\item
\item
\end{itemize}

\hypertarget{data}{%
\subsection{Data}\label{data}}

Although our methodology largely follows that presented by Määttä \&
Lessmann (\protect\hyperlink{ref-maatta}{2019}), there are some distinct
differences.

\begin{itemize}
\tightlist
\item
  Size of area
\item
  Jitter size
\item
  Probability of human-generated
\item
\end{itemize}

\hypertarget{algorithm}{%
\subsection{Algorithm}\label{algorithm}}

\hypertarget{results}{%
\section{\texorpdfstring{Results
\label{Results}}{Results }}\label{results}}

\hypertarget{discussion}{%
\section{\texorpdfstring{Discussion
\label{Conclusion}}{Discussion }}\label{discussion}}

This paper explicated on the usage of a filtering methodology proposed
by Määttä \& Lessmann (\protect\hyperlink{ref-maatta}{2019}) with which
to separate background noise from human-generated nightlight data.

Our results echo those by Määttä \& Lessmann
(\protect\hyperlink{ref-maatta}{2019}): the RF algorithm introduces
great improvements in classification accuracy and thus greater accuracy
in filtering out background noise from the `Stable Lights' product. This
allows the researcher to do away with the need for quick-and-easy type
fixes to noisy data at the lower end of the luminosity distribution.

However, it is important to note what this method does not achieve. For
instance, the `Human Lights' product does not address the issue of
blooming or oversaturation at the high end of the luminosity spectrum.
Likewise, its spatial resolution remains low in comparison to more
modern products such as the Visible Infrared Imaging Radiometer Suite
(VIIRS), and it is recommended to use these products rather than the
DMSP-OLS `Stable Lights' or `Human Lights' products if possible.

\newpage

\hypertarget{references}{%
\section*{References}\label{references}}
\addcontentsline{toc}{section}{References}

\hypertarget{refs}{}
\begin{CSLReferences}{1}{0}
\leavevmode\vadjust pre{\hypertarget{ref-maatta}{}}%
Määttä, I. \& Lessmann, C. 2019. Human lights. \emph{Remote Sensing}.
11(19):2194.

\end{CSLReferences}

\hypertarget{appendix}{%
\section*{Appendix}\label{appendix}}
\addcontentsline{toc}{section}{Appendix}

\hypertarget{appendix-a}{%
\subsection*{Appendix A}\label{appendix-a}}
\addcontentsline{toc}{subsection}{Appendix A}

\bibliography{Tex/ref}





\end{document}
